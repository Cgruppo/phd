Functional reactive programming allows developers to declaratively specify data dependency graphs \cite{wan2000functional}. Functional reactive programming has been applied to interactive graphics \cite{elliott1997functional} and robotics \cite{hudak2003arrows}. The Model View Controller paradigm is an approach to cleanly separate application operations into three classes. The \emph{model} contains the data structures representing the application state. The \emph{view} handles graphical presentation of the model to the user. The \emph{controller} translates user interactions (such as mouse clicks, key presses, or multi-touch gestures) into changes in the model \cite{krasner1988description}. The Model View Controller paradigm can be combined with functional reactive programming to enable straightworward creation of reactive systems based on data flow graphs.
\section{The Model Data Structure}
\section{Functional Reactive Change Propagation}
\section{Functional Reactive Visualizations}
\section{Functional Reactive Data Flow}
