This dissertation introduces reusable solutions for 1.) well known visualization and interaction techniques 2.) creation and sharing of visualizations with multiple linked views, and 3.) transforming existing data into a unified model for visualization and integration with other data sets. These contributions include novel data structures and algorithms for integrating and visualizing data. The overall purpose of this work is to propose new solutions to the issues surrounding data integration and visualization.

Reactive models provide the foundation for reactive visualizations. This approach allows the construction of reusable reactive flows that encapsulate elements of interactive visualizations. Using reactive flows, much of the data visualization pipeline can be implemented. Reactive flows enable representation of update flows from data and configuration through to visualization scales, axes and visual marks. In theory, any imaginable visualization technique can be implemented as a reactive visualization component. The prototypes presented in this dissertation serve as a starting point for a growing catalog of Open Source visualization modules.

Interactions such as brushing, picking and zooming can also be encapsulated using reactive models. The interactions in one visualization can be used to drive interactive changes in other visualizations. In this way visualizations with multiple linked views can be constructed. Interaction schemes for linked views include dynamic filtering, slicing, linked selection and linked probing. For example, zooming in a map can filter the regions used to filter the input to a line chart of population. This technique of using interaction for dynamic slicing can be used to build data cube exploration tools that replace small multiples with interactive linked views.

An application state model based on reactive models provides the foundation for a collaborative visual data exploration platform. Application state configuration based on JSON enables configuration of visualizations with multiple linked views. A unique runtime engine is introduced for configurable applications that dynamically loads required modules. Using this framework it is possible to author and evolve instantiations of reusable visualization components with linked views. The application state model also affords construction and navigation of history graphs supporting undo, redo, and view sharing.

The Universal Data Cube framework proposes a framework for representing data based on the data cube model. This framework is ``universal'' in that it is capable of representing aggregated summaries of any measurable quantity over limitless time and space. The Time and Space dimensions provide conceptual delineations between regions of hierarchical time and space. Measures provides windows into phenomena occuring within certain regions of time and space. In theory, all measurable quantities that summarize the Earth and human activity over time can be represented using this framework. The limited Open Source proof of concept implementation of the UDC framework is proposed as the seed for a large ongoing project aimed at curating and exposing public data.

By combining the data curated within the UDC and the proposed visualization and collaboration frameworks, the aim is to provide a complete platfom for public data visualization. The hope is that this platform will enable Web users and content creators to easily investigate any data of interest using interactive data visualization. Application areas for this technology include commercial data analysis, business intelligence, education, journalism and public policy.
